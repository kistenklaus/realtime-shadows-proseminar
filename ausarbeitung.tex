\documentclass{thesisclass}
% Based on thesisclass.cls of Timo Rohrberg, 2009
% ----------------------------------------------------------------
% Thesis - Main document
% ----------------------------------------------------------------
% No empty in-between chapters, remove this line if desired for double-page printing
\let\cleardoublepage\clearpage
%\renewcommand{\cleardoublepage}{}
%\renewcommand{\clearpage}{}

% you can add custom packages that you may need
\usepackage{amsmath}
\usepackage{amssymb}

%% ---------------------------------
%% | Information about the thesis  |
%% ---------------------------------



\newcommand{\myname}{Karl Sassie}
\newcommand{\mytitle}{Schattenverfahren auf der GPU}
\newcommand{\myinstitute}{Institut für Visualisierung und Datenanalyse,\\ Lehrstuhl für Computergrafik}

\newcommand{\reporttype}{Proseminar}                                  % Proseminar-/Seminar-/Bachelor-/Master
%\newcommand{\reviewerone}{Prof. Dr.-Ing. Carsten Dachsbacher}     % only for bachelor/master
%\newcommand{\reviewertwo}{Prof. Dr. Hartmut Prautzsch}            % only for bachelor/master
\newcommand{\advisorone}{Vincent Schüßler}
%\newcommand{\advisortwo}{?}                                      % only if applicable

\newcommand{\timestart}{28. Oktober 2022 20XX}
\newcommand{\timeend}{XX. Monat 20XX}
\newcommand{\submissiontime}{DD. MM. 20XX}

%% -------------------------------
%% |  Information for PDF file   |
%% -------------------------------
\hypersetup{
 pdfauthor={\myname},
 pdftitle={Proseminar_KarlSassie_2362789},
 pdfsubject={Schattenverfahren auf der GPU},
 pdfkeywords={Proseminar},
 colorlinks,
 linkcolor=blue,
 filecolor=red,
 urlcolor=blue,
 citecolor=blue
}

% \pdfstringdefDisableCommands{%
%   \def\\{}%
%   \def\texttt#1{<#1>}%
% }

\urlstyle{same}

\hbadness=99999  % or any number >=10000


%% ---------------------------------
%% | ToDo Marker - only for draft! |
%% ---------------------------------
% Remove this section for final version!
%
%
%%% --------------------------------
%%% | Old Marker - only for draft! |
%%% --------------------------------
%% Remove this section for final version!
%\newenvironment{deprecated}
%{\begin{color}{gray}}
%{\end{color}}


%% --------------------------------
%% | Settings for word separation |
%% --------------------------------
% Help for separation:
% In german  the following hints are additionally available:
% "- = Additional separation
% "| = Suppress ligation and possible separation (e.g. Schaf"|fell)
% "~ = Hyphenation without separation (e.g. bergauf und "~ab)
% "= = Hyphenation with separation before and after
% "" = Separation without a hyphenation (e.g. und/""oder)

% Describe separation hints here:
\hyphenation{
% Pro-to-koll-in-stan-zen
% Ma-na-ge-ment  Netz-werk-ele-men-ten
% Netz-werk Netz-werk-re-ser-vie-rung
% Netz-werk-adap-ter Fein-ju-stier-ung
% Da-ten-strom-spe-zi-fi-ka-tion Pa-ket-rumpf
% Kon-troll-in-stanz
}


%% ------------------------
%% |    Including files   |
%% ------------------------
% Only files listed here will be included!
% Userful command for partially translating the document (for bug-fixing e.g.)
\includeonly{%
titlepage,
content
}


%%%%%%%%%%%%%%%%%%%%%%%%%%%%%%%%%
%% Here, main documents begins %%
%%%%%%%%%%%%%%%%%%%%%%%%%%%%%%%%%
\begin{document}

% Remove the following line for German text
%\selectlanguage{ngerman}
%\selectlanguage{english}

\frontmatter
\pagenumbering{roman}
%% titlepage.tex
%%

% coordinates for the bg shape on the titlepage
\newcommand{\diameter}{20}
\newcommand{\xone}{-15}
\newcommand{\xtwo}{160}
\newcommand{\yone}{15}
\newcommand{\ytwo}{-253}

\newcommand{\checkfor}[2]{%
  \ifcsname#1\endcsname%
    #2
  \fi%
}

\begin{titlepage}
% bg shape
\begin{tikzpicture}[overlay]
\draw[color=gray]  
 		 (\xone mm, \yone mm)
  -- (\xtwo mm, \yone mm)
 arc (90:0:\diameter pt) 
  -- (\xtwo mm + \diameter pt , \ytwo mm) 
	-- (\xone mm + \diameter pt , \ytwo mm)
 arc (270:180:\diameter pt)
	-- (\xone mm, \yone mm);
\end{tikzpicture}
	\begin{textblock}{10}[0,0](4,2.5)
		\includegraphics[width=.3\textwidth]{logos/KITLogo_RGB.pdf}
	\end{textblock}
	\changefont{phv}{m}{n}	% helvetica	
	\vspace*{3.5cm}
	\begin{center}
		\Huge{\mytitle}
		\vspace*{2cm}\\
		\Large{
      \reporttype\iflanguage{english}{ Thesis of}{-Ausarbeitung von}
		}\\
		\vspace*{1cm}
		\huge{\myname}\\
		\vspace*{1cm}
		\Large{
      \iflanguage{english}{At the Department of Informatics}{An der Fakult\"at f\"ur Informatik}
			\\
			\myinstitute
		}\\
		\vspace*{1cm}
		\Large{\today}
	\end{center}
	\vspace*{1cm}
\Large{
\begin{center}
\begin{tabular}[ht]{l c l}
  % Gutachter sind die Professoren, die die Arbeit bewerten. 
  \checkfor{reviewerone}{\iflanguage{english}{Reviewer}{Erstgutachter}: & \hfill  & \reviewerone}\\
  \checkfor{reviewertwo}{\iflanguage{english}{Second reviewer}{Zweitgutachter}: & \hfill  & \reviewertwo}\\
  \checkfor{advisorone}{\iflanguage{english}{Advisor}{Betreuender Mitarbeiter}: & \hfill  & \advisorone}\\
  \checkfor{advisortwo}{\iflanguage{english}{Second advisor}{Zweiter betreuender Mitarbeiter}: & \hfill  & \advisortwo}\\
  % Der zweite betreuende Mitarbeiter kann weggelassen werden. 
\end{tabular}
\end{center}
}


\vspace{2cm}


\begin{textblock}{10}[0,0](4,16.8)
\tiny{ 
  \iflanguage{english}{KIT -- The Research University in the Helmholtz Association}{KIT -- Die Forschungsuniversität in der Helmholtz-Gesellschaft}
}
\end{textblock}

\begin{textblock}{10}[0,0](14,16.75)
\large{
	\textbf{www.kit.edu} 
}
\end{textblock}

\end{titlepage}

\blankpage

\chapter*{Abstract}
% Die Computer Grafik hat in den letzten Jahren sich sehr weit weiter entwickelt und was 
% vor 50 Jahren noch undenkbar war, kann heute in Echtzeit berechnet werden.
% Dahinter verbirgt sich einerseits die Weiterentwicklung der Hardware, doch auch im Software 
% Bereich sind große Forschungssprünge zu vermerken.
Ein wichtiger Teil der Computer Grafik ist die Darstellung von Schatten.
Deshalb will ich mich mit Algorithmen auseinandersetzen,
die in Echtzeit umsetzbar sind und dabei eine gute Qualität haben.
%die nicht nur eine gute Qualität haben, sondern auch in Echtzeit berechnet werden können.
Dabei werde ich mich auf eine Technik fokussieren, die 1978 von Williams vorgestellt wurde, die \glqq{}Shadow Mapping\grqq{} heißt.
Diese ist in der einfachsten Form leicht zu verstehen, doch es gibt viele 
Erweiterungen des Verfahrens die, die visuelle Qualität 
der Schatten verbessern. 
Dazu gehören Algorithmen die Schattenkanten Aliasing verhindern (zB. PCF) oder 
Algorithmen die, die relative Auflösung eines Punktes auf der Shadow Maps verändern dazu gehört, 
PSM, LiSPSM, und CSM. Es gibt sehr viel mehr solcher Algorithmen, ich will mich jedoch auf dieses kleine Sub-Set beschränken.

%
% Die Erweiterungen des \glqq{}Shadow Mapping\grqq{} Verfahrens 
% beheben meist ein visuelles Artefakt.
% Die Artefakte vollständig zu verstehen macht es leichter die einzelnen Erweiterungen zu nachzuvollziehen.
%
% Deshalb will ich diese erläutern und anschließend Anpassungen des Shadow Mapping Verfahrens vorstellen.


%% -------------------
%% |   Directories   |
%% -------------------
\tableofcontents
\blankpage

%% -----------------
%% |   Main part   |
%% -----------------
\mainmatter
\pagenumbering{arabic}
%% content.tex
%%


\chapter{Einleitung}
\label{ch:Introduction}
%motivation.
Warum sind Schatten in der Computer Grafik eigentlich so wichtig?
Schatten geben nicht nur besser aussehende Bilder, sondern geben auch wichtige Informationen über Position und Skalierung von Objekten im Raum
(Abbildung \ref{fig:why-shadows}).

\begin{figure}[H]
	\begin{center}
		\includegraphics[width=.3\textwidth]{res/img/why-shadows.jpg}
    \caption{Es ist klar zu erkennen, dass der linke Würfel höher über dem Boden schwebt als der rechte Würfel. 
    Ohne Schatten wäre es unmöglich zu entscheiden, welcher Würfel sich wieweit über dem Boden befindet.}
	  \label{fig:why-shadows}
	\end{center}
\end{figure}
%welches verfahren will ich vorstellen.
%Um den Schatten in einer Szene berechnen zu können muss zuhnächst für jeden Pixel bekannt sein, ob er im Schatten liegt.
%Dieses Problem scheint einfach. 
Ein Pixel liegt genau dann im Licht, wenn die Gerade zwischen Pixel 
und Lichtquelle von keinem anderen Objekte blockiert wird.
In einem Raytracing Renderer ist dies leicht umzusetzen. 
Um zu bestimmen, ob ein Punkt im Schatten liegt, kann ein weiterer Ray vom Oberflächenpunkt zum Licht berechnet werden.
Falls dieser ein Objekt schneidet, so liegt der Punkt im Schatten.
Raytracing ist jedoch ein sehr aufwendiges Verfahren, deshalb wird in Echtzeitberechnungen von Bildern meistens
ein rasterisierter Renderer verwendet.
In einem rasterisierten Renderer ist es jedoch nicht möglich den Schnitt eines Rays mit der Szene
zu berechnen, um zu bestimmen, ob ein Punkt im Schatten liegt.
\par
Der bekannteste Algorithmus, der Schatten in mit einem rasterisiertem Renderer 
berechnet, heißt Shadow Mapping.
In einem Raytracing Renderer wird für jeden Punkt berechnet, ob das Licht sichtbar ist.
Bei Shadow Mapping wird dies umformuliert. 
Ein Punkt liegt im Schatten genau dann, wenn der Punkt vom Licht sichtbar ist.
Dieses Verfahren bringt jedoch einige Probleme mit sich, denn um zu berechnen, ob ein Punkt von dem Licht sichtbar ist,
wird die Szene aus der Perspektive des Lichts auf eine Textur gezeichnet.
Dabei wird für jeden Pixel der Textur gespeichert, wie weit der nächstliegende Punkt von dem Licht entfernt ist.
Da es beim Abspeichern zu einer Diskretisierung kommt, 
rechnen wir in den Folgeschritten niemals mit der tatsächlichen Tiefe eines Punktes,
sondern mit einer Rundung zum nächstliegenden Diskretisierungswert.
Außerdem ist es durch die endliche Auflösung der Textur nicht möglich, die Tiefenwerte eines Punktes exakt zu speichern,
sondern nur die Tiefenwerte eines Bereiches, der von einem Pixel auf der Textur beschrieben wird.
\par
Trotz der vielen Probleme wird Shadow Mapping in vielen Echtzeit Applikationen verwendet.
Deshalb werde ich mich im weiteren mit dem Algorithmus auseinandersetzen und dabei die Probleme, die dabei auftreten untersuchen.
Dabei werde ich zunächst auf die Wert Diskretisierung in der Tiefen-Textur eingehen.
Dies führt zu einem Artefakt, dass Selbstverschattung genannt wird. 
Um dieses Artefakt zu verhindern, gibt es viele Möglichkeiten. 
Ich will mich jedoch mit Heuristischen Biasing Methoden und Second Depth Shadow Mapping beschäftigen.
Anschließend werde ich auf projektives und perspektivisches Aliasing eingehen. 
Diese Artefakte entstehen durch die limitierte Texturauflösung der Shadow Map.
Um diese Artefakte zu beheben, gibt es eine Überzahl an Verfahren. Ich werde mich jedoch auf PSM, LiSPSM, CSM und PCF beschränken.
Diese Verfahren bieten eine gute Grundlage für das Verständnis weiterer komplexerer Verfahren.
Am Ende will ich in einem Fazit eine echtzeitfähige Kombination der behandelten Verfahren vorstellen.


% thematische einordnung (was auch immer has heißen soll)
% Kapitel gliederung erklären.

\chapter{Grundlagen der Rasterisierung}
\section{Koordinatensysteme}
\label{section:coords}
% hier wäre ein paar zitate schön die das ganze gut auf den punkt bringen.
Beim rasterisierten Rendern zeichnen wir Objekte, die aus Oberflächen (Dreiecken) bestehen.
Oberflächen bestehen aus Punkten, die die Position der Oberfläche relativ zum Objektmittelpunkt beschreiben.
Diese Koordinaten, die relativ zum Objekt gegeben sind, befinden sich in einem Koordinatensystem, das \textbf{Object Space} heißt.
Um ein Objekt in der Welt verschieben zu können müssen alle Punkte eines Objekts (in Object Space) mit einer linearen Transformation
in \textbf{World Space} transformiert werden.
Die Transformation besteht aus Skalierung, Translation und Rotation und wird meistens als 4x4 Matrix mit homogenen Koordinaten dargestellt.
Das Konzept einer Kamera lässt sich analog einführen.
Dabei werden alle Punkte im World Space mit einer linearen Transformtion in \textbf{View Space} transformiert.
In View Space sind alle Punkte relativ zu der Kamera.
In einer weiteren Transformation wird nun die Art der Projektion der Kamera definiert. 
Dabei wird ein Punkt im View Space in \textbf{Clip Space} transformiert.
Die Transformation bildet alle Punkte, die sichtbar sind, auf einen Punkt innerhalb eines Würfels der Größe 2x2x2 um den Ursprung ab.
Alle Oberflächen, die außerhalb des Würfels liegen sind von der Kamera nicht sichtbar und werden nicht gezeichnet.
Aktuell haben wir nur Transformationen der Form $\mathbb{R}^3 \rightarrow \mathbb{R}^3$ betrachtet.
Um die Punkte auf dem Bildschirm darzustellen müssen wir eine Transformation der Form 
$\mathbb{R}^3 \rightarrow [0,1080]\times[0,1920]$ (Unter der Annahme, das der Bildschirm FullHD ist) anwenden.
Eine solche Transformation wird von der Renderpipeline implizit durchgeführt. 
Dabei werden alle Punkte in Clip Space 
auf die XY-Ebene projiziert und die Z-Werte werden als Tiefenwert verwendet. 
Die Transformation transformiert somit Punkte von Clip Space in \textbf{Screen Space}.

\section{Projektion}
In Abschnitt \ref{section:coords} wurde bereits auf die einzelnen Koordinatensysteme und Transformationen eingegangen.
Für die Kamera Projektion ist jedoch eine genaueres Verständis nötig.
Eine Kamera Projektion ist eine Abbildung von View Space in Clip Space.
Im Allgemeinen lassen sich alle Kamera Projektionen mit einem Frustum (Kegelstumpf) eindeutig charakterisieren.
Ein Kamera Frustum beschreibt einen Raum im View Space, 
der durch die Projektion auf einen Würfel im Clip Space abgebildet wird.
Eine Oberfläche des Frustums, die Near Plane, ist die Oberfläche auf die projiziert wird.
Die entgegengesetzte Oberfläche heißt Far Plane. 
Die Far Plane beschreibt die Punkte, die am weitesten von der Kamera entfernt sind und noch sichtbar sind.
Eine Gerade wird von einem Punkt auf der Near Plane zu einem entsprechenden Punkt auf der Far Plane
durch die Projektion auf eine Gerade projiziert, die orthogonal zu der XY-Ebene liegt.
Dadurch wird klar, dass die Projektionstransformation, die Art der Projektion der Kamera definiert.
\par
Projektionstransformationen werden oft mit einer 4x4 Matrix implementiert.
Für eine orthogonale Projektion ist dies nicht nötig, da hier von $\mathbb{R}^3$ nach $\mathbb{R}^3$ abgebildet wird.
Bei perspektivischen Projektionen wird die W-Komponente verwendet, um die Perspektive zu implementieren.
Dabei werden nach der Matrix Multiplikation die XYZ-Komponenten durch die W-Komponente des Ergebnisvektors geteilt.
Dabei ist zu bemerken, dass die Z-Koordinate ebenfalls durch die W-Komponente geteilt wird. 
Dies ist im allgemeinen sinnvoll, denn es führt zu einer nicht linearen Verteilung der Tiefe, wodurch die Tiefenauflösung
nahe bei der Kamera höher ist.
\par
Es ist zu bedenken, dass ebenfalls durch die Z-Komponente durch die W-Komponente geteilt wird, wodurch die Verteilung der Tiefenwerte direkt mit der 
Art der Projektion zusammenhängt. Beispielsweise haben orthogonale Projektionen oft eine lineare Verteilung der Tiefenwerte,
wobei perspektivische Projektionen oft eine nicht lineare Verteilung mit sich bringen.

\section{Framebuffer}
Zum Zeichnen der Szene werden die Farb- und Tiefenwerte der Pixel in einem Framebuffer gespeichert.
Tiefenwerte können dabei unterschiedlich kodiert werden. 
Oft werden unsigned 32-bit Integer verwendet.
Da die Werte diskret gespeichert werden, enthält der Framebuffer nach dem Rendern nicht die tatsächlichen Farb- und Tiefenwerte.
Sondern eine Rundung auf den nächst gelegenen Wert.
Für die Farbwerte ist dieses Verhalten kein Problem, da der Unterschied zwischen 2 Farbwerten, die nebeneinander liegen, nicht erkennbar ist.
Dieses Verhalten führt jedoch bei Tiefenwerten dazu, dass die Tiefenwerte des Framebuffers niemals die tatsächliche Tiefe beschreiben.
Dieses Umstand ist im Allgemeinen durch die nicht lineare Verteilung der Tiefenwerte, zwischen Near und Far Plane, kein Problem.
Im Weiteren werden wir jedoch sehen, dass dies durchaus zu Problemen führen kann, wenn die Szene aus einer anderen Perspektive gezeichnet wird 
und die Ergebnisse in einem weiteren Zeichenschritt verwendet werden.

\chapter{Shadow Mapping} %that title blows.
\label{section:shadow-mapping}
\section{Uniform Shadow Mapping}
\label{section:uniform-shadow-mapping}
Uniform Shadow Mapping \cite{Williams1978} ist ein Algorithmus zu Berechnung von Schatten 
mit einem rasterisiertem Renderer. 
Es ist das Verfahren, das in der Echtzeitberechnung am meisten zum Einsatz kommt, 
da es in einem rasterisiertem Renderer implementierbar ist.
\par
Wie bereits im Abschnitt \ref{ch:Introduction} erwähnt ist es zur Berechnung von Schatten nötig, festzustellen,
ob ein anderes Objekt das Licht an einem Punkt blockiert.
Diese Berechnung für jeden Pixel direkt auszuführen ist in einem rasterisiertem 
Renderer nicht möglich.
Beim Shadow Mapping Verfahren formulieren wir die Frage um und betrachten 
die äquivalente Frage:
\textbf{Ist der Punkt von der Perspektive des Lichtes sichtbar?}
Wenn ja, dann wird das Licht von keinem Objekte blockiert.
Dieses Umformulieren der Fragestellung hat den Vorteil, dass sie sich in
einem rasterisierten Renderer berechnen lässt, indem die Szene aus der Perspektive des Lichtes gezeichnet wird.
\par
\textbf{Der Shadow Mapping Algorithmus lässt sich in 2 Schritte 
unterteilen:}
\begin{enumerate}
  \item Im ersten Schritt wird ein Koordinatensystem (\textbf{Light View Space}) konstruiert, 
    das die Szene relativ zum Licht beschreibt.
    Anschließend wird auf dieses Koordinatensystem eine Projektionstransformtion angewendet.
    Das neue Koordinatensystem heißt \textbf{Light Clip Space}.
    Die Szene wird nun mit dem Light Clip Space Koordinatensystem gezeichnet.
    Dabei muss nur der Tiefenwert für jeden Pixel gespeichert werden.
    Die Textur mit den Tiefenwerten heißt \textbf{Shadow Map}.
  \item Im zweiten Schritt wird die Szene aus der Perspektive der Kamera gezeichnet.
    Gleichzeitig zu der Transformation der Punkte von Object Space zu Clip Space
    werden die Punkte ebenfalls in Light Clip Space transformiert.
    Nun kann der Abstand des Punktes zum Licht 
    aus der Z-Komponente des Punktes im Light Clip Space Koordinatensystems abgelesen werden.
    Mit der X- und Y-Komponente kann bestimmt werden, auf welchen Pixel in der Shadow Map der Punkt in Schritt 1 abgebildet wurde.
    Anschließend kann durch einen Vergleich bestimmt werden, ob der Punkt im Schatten liegt.
    Wenn der Tiefenwert des Pixels in Shadow Map kleiner ist als die Z-Komponente des Punktes im Light Clip Space,
    so ist der Punkt von der Perspektive des Lichtes nicht sichtbar und liegt somit im Schatten.
\end{enumerate}
\begin{figure}
	\begin{center}
		\includegraphics[width=1\textwidth]{res/img/shadow-mapping/shadow_map_principle.png}
    \caption{[Links] die Perspektive der Kamera mit eingezeichneter Lichtposition. 
    [Rechts] Die Tiefenwerte der Szene aus der Perspektive des Lichtes. 
    Diese Textur wird Shadow Map genannt.}
    \label{fig:shadow-map}
	\end{center}
\end{figure}
%arten von lichtquellen.
\section{Arten von Lichtquellen}
\label{section:types-of-sources}
Es gibt unterschiedliche Arten von Lichtern die sich alle mit Shadow Mapping darstellen lassen.
Das direktionales Licht ist ein Licht, bei dem alle Lichtstrahlen parallel verlaufen.
Dies lässt sich beim Shadow Mapping umsetzen, indem als Projektionsmatrix des Lichtes eine
orthogonale Projektion verwendet wird.
Analog lässt sich ein Spot Light implementieren, indem als Projektion eine perspektivische Projektion verwendet.
Punkt-Lichter können etwas komplizierter sein, doch wir wollen Punkt-Lichter hier einfach als 6 Spotlights betrachten
die jeweils 90° abdecken. Dabei wird die Shadow Map in einer Würfel-Textur mit 6 Seiten gespeichert.
Der Shadow Mapping Algorithmus verläuft für alle Lichter identisch, nur die Wahl der Projektion wird verändert.

\section{Überblick der Diskretisierungs Probleme}
\label{section:problem-overview}
Beinahe alle Probleme des Shadow Mapping Verfahren entstehen durch die Diskretisierung,
die gezwungenermaßen in Computern auftritt. 
Dabei tritt Diskretisierung in mehreren Dimensionen auf.
\par
Die erste Form der Diskretisierung, die zu Fehlern führt, ist die Werte-Diskretisierung 
in der Shadow Map. 
Beim Vergleich, ob ein Pixel im Schatten liegt oder von einer Lichtquelle beleuchtet wird,
kommt es zur falschen Klassifikation. Dieser Fehler entsteht durch Runden der Tiefenwerte in der Shadow Map und
durch die numerische Instabilität der Lichttransformation.
Dies führt zu einem deutlich sichtbaren visuellen Artefakt, das Selbestverschattung genannt wird.
Selbestverschattung wird in Abschnitt \ref{section:self-shadow} genauer erklärt.
\par
Die zweite Dimension, in der Diskretisierungsfehler auftreten, ist die Pixel Auflösung der Shadow Map.
Diskretisierungsfehler, die dadurch auftreten, erzeugen Aliasing Artefakte.
Eine Möglichkeit diese zu verbessern funktioniert analog zu Geometrie Antialiasing. 
Dabei kann jedoch nicht die Shadow Map selbst gefiltert werden, sondern es werden die Vergleichsergebnisse gefiltert.
Dies führt dazu, dass die Schattenkanten geglättet werden. 
Eine mögliche Implementierung hiervon heißt Percentage Closer Filtering
und wird in Abschnitt \ref{section:pcf} behandelt.
\par
Um andere Methoden zur Verbesserung des Aliasing Problems zu betrachten ist eine genauere Klassifikation des Aliasing Problems 
nötig. 
Dabei wird Aliasing in 2 Kategorien eingeteilt:
\begin{itemize}
  \item \textbf{Perspektive Aliasing} entsteht dadurch, dass Oberflächen, die nahe bei der Kamera sind groß auf dem 
    Bildschirm dargestellt werden, womöglich aber klein in der Shadow Map sind.
    Falls eine Oberfläche nah bei der Kamera ist, so folgt daraus, dass sie groß im Clip Space ist.
    Leider folgt daraus nicht, dass die Oberfläche ebenfalls groß im Light Clip Space ist.
    Daraus folgt, dass 
    Oberflächen, die nahe bei der Kamera sind, groß im Clip Space sind, aber teilweise klein im Light Clip Space,
    dadurch sind sie groß auf dem Bildschirm sind, aber nur sehr klein in der Shadow Map. 
    Die Schattenauflösung an solchen Oberflächen ist besonders schlecht.
    Der Effekt von Perspektive Aliasing ist maximal, wenn das Licht direkt in die Kamera leuchtet, denn hier 
    sind Oberflächen nahe bei der Kamera im Clip Space groß aber im Light Clip Space klein.
    Hier werden Oberflächen nahe bei der Kamera wenigen Pixeln auf der Shadow Map zugeordnet. 
    Dieses Szenario nennt man Duelling Frusta. 
    Analog ist der Effekt minimal, wenn die Lichtrichtung gleich der Sichtrichtung ist.
    \begin{figure}[H]
      \begin{center}
        \includegraphics[width=1\textwidth]{res/img/perspective-aliasing1/image.jpg}
        \caption{}
        \label{fig:perspective-aliasing}
      \end{center}
    \end{figure}
  \item \textbf{Projektives Aliasing} entsteht, wenn zwischen Pixeln in der Shadow Map und Pixeln 
    im Screen Space kein 1:1 Verhältnis existiert. Dieser Effekt ist besonders stark bei Oberflächen
    mit Normalen orthogonal zu der Lichtrichtung. Diese Oberflächen nehmen auf der Shadow Map nur sehr wenig 
    Pixel ein, wogegen sie im Screen Space sehr groß sein können. 
    \begin{figure}[H]
      \begin{center}
        \includegraphics[width=1\textwidth]{res/img/projective-aliasing/image.jpg}
        \caption{}
        \label{fig:projective-aliasing}
      \end{center}
    \end{figure}
\end{itemize}


\section{Selbestverschattung}
\label{section:self-shadow}
%erläuterung des Artefakts (mit Bild)
\begin{figure}[H]
	\begin{center}
    \includegraphics[width=1\textwidth]{res/img/shadow-acne1/image.jpg}
    \caption{Links\hspace{6pt}: Szene mit Selbstverschattung. 
         \newline Rechts: Szene ohne Selbstverschattung.} 
	  \label{fig:shadow-acne-artefact}
	\end{center}
\end{figure}
Selbstverschattung (Abbildung: \ref{fig:shadow-acne-artefact}) ist ein visuelles Artefakt.
Es tritt auf, wenn ein Punkt in der Shadow Map näher ist als tatsächlich, wodurch der Punkt sich selbst beschattet.
Dies kann an folgen Gründen liegen.
\begin{enumerate}
  \item Die Rundung der Tiefenwerte kann zu Selbestverschattung führen, dabei kann entweder der Tiefenwert in der Shadow Map
    abgerundet oder die Z-Komponente des Punktes in Light Clip Space aufgerundet worden sein.
  \item In der Shadow Map kann nicht die Tiefe eines Bereichs
    gespeichert werden, sondern nur die Tiefe eines Punktes.
    Bei Oberflächen die parallel zur Lichtrichtung liegen und klein auf der Shadow Map dargestellt tritt dieses Problem verstärkt auf.
\end{enumerate}


\begin{figure}[H]
	\begin{center}
    \includegraphics[width=1\textwidth]{res/img/shadow-acne3/image.png}
    \caption{Die Box ist ein beleuchtetes Objekt. 
      Die zackige gelbe Linie beschreibt die Tiefenwerte die tatsächlich in der Shadow Map gespeichert sind.
      Hier sind 2 Diskretisierungen zu bemerken; 
      die Breite der Shadow Map Pixel und der Höhenunterschied der Pixel (Tiefenauflösung).
      Der schwarze Teil der Box Oberfläche liegt im Licht, 
      da hier der Tiefenwert in der Shadow Map größer ist als der Abstand der Oberfläche zum Licht.
      Analog liegt der gelbe Teil der Box Oberfläche im Schatten.
    }
	  \label{fig:shadow-acne-illu}
	\end{center}
\end{figure}

Selbstverschattung kann durch einen Bias-Wert verringert werden.
Dabei wird vor dem Vergleich ein kleiner Wert (der Bias), auf den Tiefenwert in der Shadow Map addiert.
Bildlich kann man sich vorstellen, dass die gelbe gezackte Linie
(aus Abbildung: \ref{fig:shadow-acne-illu}) in die Lichtrichtung verschoben wird.
Es tritt keine Selbstverschattung mehr auf, wenn die komplette gelbe Linie unterhalb der Oberfläche liegt.
\par
Ein Konstanter Bias kann dabei helfen die Rundungsfehler zu verhindern.
Jedoch tritt weiterhin Selbstverschattung an Oberflächen auf, die fast parallel zur Lichtrichtung liegen.
An diesen Oberflächen kann Selbstverschattung nur durch einen sehr großen Bias-Wert verhindert werden.
Dies führt jedoch zu einem Artefakt, dass Peter Panning heißt (Abbildung: \ref{fig:peter-panning}).
Dadurch, dass wir die Tiefenwerte der Shadow Map in Richtung des Lichtes mit einem Bias-Wert verschieben,
verschieben wir auch die Schatten selbst.
Dieses Artefakt ist irritierend, den wie in Abschnitt \ref{ch:Introduction} erläutert, geben uns Schatten Informationen
über Skalierung und Position. Wenn Schatten, durch Peter Panning verschoben werden, dann führt das zu einer irreführenden Darstellung der Szene.

\begin{figure}[H]
	\begin{center}
    \includegraphics[width=1\textwidth]{res/img/peter-panning1/image.jpg}
    \caption{Links\hspace{6pt}: Szene mit Peter Panning Artefakt. Es führt oft dazu das es so scheint, als würden Objekte schweben.
        \newline Rechts: Szene ohne Peter Panning Artefakt.}
	  \label{fig:peter-panning}
	\end{center}
\end{figure}

%einfache erläuterung von constant depth-bias.
%erläuterung von Peter panning (mit Bild)

\subsection{Heuristisches Biasing}
\label{section:heuristic-biasing}
Wie wir im letzten Kapitel gesehen haben, kommt es ohne einen Bias zu Selbestverschattung
und falls ein zu großer Bias verwendet wird, kommt es zu Peter Panning.
Es sei außerdem zu bemerken, dass bei Oberflächen, die orthogonal zur Lichtrichtung liegen, 
ein nur sehr geringer Bias benötigt wird, wobei für Oberflächen, die parallel zur Lichtrichtung 
liegen ein großer Bias nötig ist.
Diese Beobachtung lässt sich ausnutzen, indem der Bias Wert proportional zum sinus des Winkels zwischen
der Lichtrichtung und der Oberflächennormalen gewählt wird.
Anstatt den Winkel direkt auszurechnen, lässt sich dies effektiv mit dem Standard-Skalarprodukt berechnen.
Ein solcher Bias wird Slope-Scale-Bias genannt.
Eine alternative Möglichkeit den proportionalen Bias einzusetzen heißt Normal Offset Shadow Mapping.
Dabei wird die Oberfläche anstatt in die Richtung des Lichtes in die Richtung des Normalen Vektors 
verschoben.
\\
Beide dieser Verfahren verwenden dieselbe Heuristik wobei die Stärke des Bias 
proportional zu $1 - \overrightarrow{l} * \overrightarrow{n}$, wobei $\overrightarrow{l}$ 
die Lichtrichtung und $\overrightarrow{n}$ der Normalenvektor der Oberfläche ist.
Es sollte erwähnt werden, dass für Oberflächen die orthogonal zu der Lichtrichtung liegen
der heuristische Bias 0 ist.
Deshalb werden diese Biasing-Methoden oft mit einem konstanten Bias kombiniert.
\par
Das große Problem mit heuristischen Biasing-Methoden ist, dass 
sie für jede Szene neu konfiguriert werden müssen.
Denn falls der Bias zu groß ist, kommt es zu Peter-Panning und falls er zu klein gewählt 
wird, kann es zu Selbstverschattung kommen.
Um die Konfiguration leichter zu machen werden oft Kombinationen 
von unterschiedlichen Biasing-Methoden verwendet wie zum Beispiel : \\
Slope-Scale-Bias + Normal-Offset-Shadow-Mapping + Konstanter Bias.
%TODO WHY DOES NORMAL OFFSET SHADOW MAPPING WORK.
\subsection{Alternative Techniken}
\label{section:sdsm}
Eine Alternative zu einem Bias Wert heißt Second Depth Shadow Mapping.
Dieses funktioniert jedoch nur für Szenen mit solider Geometrie.
Dabei wird in der Shadow Map anstatt der Geometrie, die am nächsten zum 
Licht ist, die Geometrie abgespeichert die am zweit nächsten zum Licht ist.
Dies lässt sich effektiv implementieren, indem anstatt von Back Face Culling, Front Face Culling verwendet wird.
Dadurch kommt es zu keiner Selbestverschattung mehr, da die Oberfläche, die am nächsten an der 
Lichtquelle liegt, immer im Licht ist. 
Denn beim Vergleich der Tiefenwerte werden keine zwei Werte verglichen, die nah beieinander liegen.
Es könnte jedoch Selbestverschattung auf der Rückseite der Geometrie auftreten.
Da die Geometrie solide ist, kann für Oberflächen, die vom Licht abgewandt sind direkt 
angenommen werden, dass sie im Schatten liegen.
Außerdem tritt kein Peter-Panning auf, da kein Bias verwendet wird.
Doch wie bereits erwähnt erfordert Second Depth Shadow Mapping, dass die
Geometrie der Szene solide ist.
Doch auch mit solider Geometrie kann Selbestverschattung auftreten.
Wenn die Geometrie sehr dünn ist, ist der Tiefen unterschied zwischen Front und Back Face zu gering, 
wodurch es beim Vergleich der Tiefenwerte wieder zu Selbstverschattung kommt.

\section{Schattenkanten Antialiasing}
\label{section:pcf}
In der Implementierung des Shadow Mapping Algorithmus aus Abschnitt \ref{section:uniform-shadow-mapping} wird für jeden Oberflächenpunkt 
binär klassifiziert, ob der Punkt im Schatten liegt. Dies führt jedoch bei geringer Shadow Map Auflösung zu Aliasing.
Eine Möglichkeit diesen Effekt zu verringern, ist Schattenkanten zu glätten.
Dabei ist zu beachten, dass bekannte Antialiasing Algorithmen wie MSAA oder SSAA nicht einsetzbar sind.
Wenn auf die Shadow Map direkt ein solcher Filter angewandt wird, so filtern wir die Tiefenwerte.
Mit der gefilterten Shadow Map lässt sich nun keine Schattenberechnung mehr durchführen, da alle Tiefenwerte 
verfälscht sind.
Anstelle der Tiefenwerte, muss die Klassifikation, ob ein Punkt im Licht ist, gefiltert werden.
\par
Percentage Closer Filtering(PCF) \cite{PCF1987} ist ein einfacher Algorithmus der Schattenkanten glättet.
Dabei wird bei PCF nicht die Shadow Map selbst gefiltert, sondern die Klassifikation.
Der Algorithmus wird für jeden Oberflächenpunkt ausgeführt und lässt sich in 3 Schritte unterteilen:
\begin{enumerate}
  \item Wie zuvor wird die Position des Punktes in Light Clip Space bestimmt.
  \item Der Vergleich der Tiefenwerte wird nun jedoch für alle Pixel der Shadow Map in einem $N\times{}N$ Feld durchgeführt.
  \item Der Durchschnitt der binären Vergleiche stellt den Anteil des Lichtes dar, das den Punkt erreicht.
\end{enumerate}
\par
Der Parameter $N$ beschreibt dabei die Auflösung der Schattenglättung.
\par
Dadurch, dass bei PCF für jeden Punkt $N^2$ Shadow Map Pixel gelesen werden müssen, gehört PCF zu den 
langsameren Antialiasing Algorithmen.
In \cite{PCF1987} wird dieses Problem bemerkt und es werden stochastische Sampling Methoden vorstellt.
Dabei werden die Pixel, die für den Vergleich verwendet werden, stochastisch mittels Monte Carlo Sampling gewählt.
\par
Es ist jedoch zu bemerken, dass selbst mit Monte Carlo Sampling, PCF kein effektiver Algorithmus ist und
in dieser Ausarbeitung nur aus Gründen der Vollständigkeit enthalten ist. 
Die Schattenglättung in der Echtzeitberechnung ist ein großes Thema mit vielen Algorithmen. 
Dazu gehört Variance Shadow Mapping, Exponential Shadow Mapping, Logartihmic Shadow Mapping,
Percentage Closer Soft Shadows und noch einige mehr auf die ich hier nicht weiter eingehen werde.

\section{Bestimmen eines engen Licht Frustums}
\label{section:fit-the-frustum}
Die Auflösung der Shadow Map steht im direkten Verhältnis zur Größe des Frustum.
Die Tiefe des Frustums (Abstand zwischen Near und Far Plane) ist relativ zu der Tiefenauflösung der Shadow Map.
Der Bereich, der von einem Pixel der Shadow Map beschrieben wird, ist relativ zu der Breite und Höhe des Frustum.
Ist das Frustum groß, so beschreibt ein Pixel der Shadow Map einen großen Bereich im World Space.
\par
Daraus folgt, dass eine Möglichkeit die Qualität der Schatten zu verbessern, das Bestimmen eines möglichst kleinem
Licht Frustum ist. Dabei sollte für das Licht Frustum gelten:
\begin{enumerate}
  \item Alle Oberflächenpunkte die Schatten werfen können, sind in dem Licht Frustum enthalten.
  \item Die Form des Licht Frustum bleibt gleich.
  \item Die Größe des Licht Frustum ist minimal.
\end{enumerate}
\begin{figure}[H]
	\begin{center}
    \includegraphics[width=0.6\textwidth]{res/img/fit-the-frustum.jpg}
    \caption{}
	  \label{fig:fit-the-frustum}
	\end{center}
\end{figure}
Eine einfache Implementierung funktioniert wie folgt:
\begin{enumerate}
  \item Transformiere alle Punkte des 2x2x2 Würfels im Clip Space mit der Inversen Kamera Projektion
    in View Space. Die Eckpunkte des Würfels werden dabei auf die Eckpunkte des Kamera Frustum abgebildet.
  \item Transformiere das Kamera Frustum in Light View Space.
  \item Bestimme die maximale und minimale Z-Komponente des Kamera Frustum in Light View Space.
  \item Die Far Plane ist nun durch die maximale Z-Komponente definiert und die Near 
    Plane durch die minimale. Dadurch, dass Punkte die außerhalb des Kamera Frustums liegen ebenfalls Schatten werfen können,
    wird die Near Plane oft nach hinten verschoben.
\end{enumerate}
Hier wird sichergestellt, dass alle Punkte die im Kamera Frustum enthalten sind, ebenfalls im
Licht Frustum enthalten sind. Dies ist unnötig, den nicht alle Punkte im Kamera Frustum müssen auch Schatten werfen.
Ein minimaleres Frustum kann bestimmt werden, indem zuerst eine Bounding Box um alle Objekte berechnet wird,
um anschließend das Frustum so zu wählen, 
dass der Schnitt zwischen Objekt Bounding Boxes und dem Kamera Frustum vollständig im
Licht Frustum enthalten ist.
\par
Bei Lichtern mit einer perspektivischen Projektion, kann nur die Near und Far Plane angepasst werden, da sich sonst die Form der Projektion
ändert. 
Dadurch kommt es nur zu einer Verbesserung der Tiefenauflösung.
Für Lichter mit orthogonaler Projektion kann ebenfalls die effektive Pixel Auflösung der Shadow Map verbessert werden.
Für solche Lichter kann das Kamera Frustum, wie bei Lichtern mit perspektivischer Projektion, in Light View Space 
transformiert werden, um anschließend eine Bounding Box um das Kamera Frustum zu bestimmen. 
Für große direktionale Lichter wie die Sonne ist dies besonders wichtig, da sonst die gesamte Szene in der Shadow Map enthalten wäre,
wodurch die Auflösung der Shadow Map extrem gering ausfallen würde.

\section{Perspective Shadow Mapping}
\label{section:psm}
Eine Möglichkeit, um perspektivisches Aliasing zu verhindern, ist das Verzerren 
der Shadow Map. 
Dabei wird das Light Clip Space Koordinatensystem so verzerrt, dass Oberflächen die nahe an der Kamera liegen in dem
Lichtkoordinatensystem größer sind und somit eine höhere Schatten Auflösung haben.
\newline
Eine mögliche Implementierung davon wird auch Perspective Shadow Mapping (PSM) \cite{PSM2002} genannt.
Die Idee von PSM ist es, zuerst die Szene in Clip Space zu transformieren, um anschließend die Schattenberechnung 
im Clip Space durchzuführen.
Es gilt, dass Objekte die sich nahe an der Kamera befinden, größer im Clip Space sind.
Somit sind sie ebenfalls größer auf der Shadow Map, wodurch weniger perspektivisches 
Aliasing entsteht.
PSM scheint eine fantastische Möglichkeit zu sein, um perspektivisches Aliasing zu verhindern.
Jedoch kommt es zu folgenden Problemen:
\begin{itemize}
  \item Dadurch, dass die Schattenberechnung in Clip Space umgesetzt wird,
    müssen Direktionale Lichter als Punkt Lichter und Punkt Lichter teilweise als Direktionale Lichter interpretiert werden.
    Das entspricht nicht dem menschlichen Sehgewohnheiten und kann verwirrend wirken.
  \item Es tritt weiterhin Projektives Aliasing auf.
  \item Die Verbesserung der Schattenauflösung nahe bei der Kamera führt zu einer starken Reduzierung der Schattenauflösung
    für Objekte, die weiter von der Kamera entfernt sind.
\end{itemize}
\begin{figure}
	\begin{center}
    \includegraphics[width=0.8\textwidth]{res/img/psm_dir_lights.png}
    \caption{Direktionale Lichter in World Space (oben) werden Punkt Lichter im Clip Space (unten).}
	  \label{fig:psm-dir-lights}
	\end{center}
\end{figure}
\begin{figure}
	\begin{center}
    \includegraphics[width=0.8\textwidth]{res/img/psm_point_lights.png}
    \caption{[Links]: Ein Punkt Licht vor dem Betrachter bleibt ein Punkt Licht. \newline
      [Mitte]: Ein Punkt Licht hinter dem Betrachter wird ein
    invertiertes Punkt Licht (alle Lichtstrahlen gehen zu einem Punkt).}
	  \label{fig:psm-point-lights}
	\end{center}
\end{figure}
Perspektivische Transformationen des Kamera Frustum sind sinnvoll für das Bekämpfen von perspektivischem Aliasing,
jedoch ist es nicht nötig, dass diese, wie bei PSM, an das View Frustum gebunden ist.
Bei Light Space Perspektivische Shadow Mapping (LiSPSM) \cite{LiPSM2004} wird die perspektivische Transformation so
gewählt, dass nur Geraden parallel zu der Shadow Map verzerrt werden. 
Dies führt dazu, dass es weiterhin zu weniger perspektivischem Aliasing kommt.
Jedoch bleiben jetzt direktionale Lichter, direktionale Lichter.
Das bedeutet, dass LiSPSM oft leichter zu implementieren ist, da nicht so viele Spezialfälle betrachtet werden müssen.
\par
Die Stärke der Verzerrungsmatrix ist durch einen Parameter $n$ definiert.
In dem gleichen Paper in dem LiSPSM vorgestellt wurde, wird ebenfalls eine Wahl für $n$ vorgestellt,
durch die projektives Aliasing in den meisten Fällen minimiert wird: $ n_{opt} = \frac{near + \sqrt{near * far}}{sin(\gamma)} $. 
Wobei $\gamma{}$ den Winkel zwischen Lichtrichtung und Kamera Sichtrichtung beschreibt.
\par
Ebenfalls wird in \cite{LiPSM2004} erwähnt, dass eine logarithmisch parameterisierte Shadow Map, eine optimale Lösung wäre.
Jedoch lässt sich dies nicht leicht in Hardware implementieren.
\section{Cascaded Shadow Mapping}
\label{section:csm}
Cascaded Shadow Mapping(CSM) \cite{Engel2007}\cite{Zhang2006} ist ein weiteres Verfahren, um perspektivisches Aliasing zu vermeiden.
Im Vergleich zu LiSPSM wird der Raum bei CSM nicht verzerrt, sondern in $N$ Räume partitioniert.
\par
Der Algorithmus lässt sich in folge Schritte unterteilen:
\begin{enumerate}
  \item Partitioniere das View Frustum in N Räume. 
    Teile dafür das View Frustum an N Oberflächen die orthogonal zu der Near und Far Plane liegen.
    Die Schnittebenen werden dabei exponentiel gewählt, dies folgt aus den Herleitungen die in LiSPSM geführt wurden: 
    $z_{i} = N * (far / near)^{i/N}$
  \item Berechne eine enge Projektion [\ref{section:fit-the-frustum}] um jede Partition des View Frustums.
    Zeichne anschließend die Szene mit den $N$ berechneten Projektionen.
    Die $N$ erzeugeten Shadow Maps heißen Kaskaden.
    Jede Kaskade hat dieselbe Auflösung, beschreibt jedoch einen unterschiedlich großen Bereich.
    Kaskaden, die weiter von der Kamera entfernt sind, sind größer und die Schatten, die durch diese Kaskaden erzeugt werden, haben somit eine 
    geringere Qualität.

  \item Zeichne die Szene aus der Kamera Perspektive. 
    Die Schattenberechnung muss dabei leicht angepasst werden.
    Beim Vergleich im Shadow Mapping Algorithmus muss jetzt der Tiefenwert mit allen Kaskaden verglichen werden.
\end{enumerate}
Es ist zu bemerken, dass CSM eine Diskretisierung von LiSPSM ist. 
Denn falls $N\rightarrow\infty$ sind beide Verfahren identisch. 
Durch die Diskretisierung kommt es dazu, dass der Übergang zwischen Kaskaden oft sichtbar ist.
Es ist möglich LiSPSM zu verwenden, um die einzelnen Kaskaden zu zeichnen. 
Doch dies führt dazu, dass die Schattenberechnung wieder
in Clip Space durchgeführt wird. 
Eine andere Möglichkeit ohne die Form des Lichtfrustums zu verändern, basiert darauf die Kaskaden zu überlappen,
wodurch Schattenqualität verloren geht. 
Doch da Übergang zwischen den Kaskaden nicht mehr sichtbar ist,
führt es das zu einer Steigerung der Schattenqualität.
\par
Der große Vorteil von CSM ist, dass die Berechnung der Schatten nicht in Clip Space durchgeführt wird. 
Dadurch können Punktlichter als Punktlichter und direktionale Lichter als direktionale Lichter behandelt werden.
CSM eignet sich außerdem für große Lichtquellen wie beispielsweise Sonnen, die die gesamte Szene beleuchten.
Solche Lichter benötigen eine hohe Auflösung. 
Da die Auflösung in CSM auf $N$ Texturen aufgeteilt wird, können hierfür beispielsweise 4x4000x4000 Texturen verwendet werden. 
Diese Auflösung ist mit LiSPSM nicht möglich, da viele Grafikkarten keine 16000x16000 Texturen unterstützen.
Es sollte jedoch bei CSM nicht vergessen werden, dass es perspektivisches Aliasing nicht so gut entfernt, wie 
LiSPSM, da CSM eine Diskretisierung von LiSPSM ist.
In der Praxis ist die Diskretisierung von CSM jedoch selten erkennbar, weshalb CSM verbreiteter wie LiSPSM ist \cite{Survey2011}.

\chapter{Fazit}
Der Shadow Mapping Algorithmus ist eine gute Möglichkeit um Schatten in Echtzeit zu berechnen.
Es ist jedoch wichtig, die zugrunde liegenden Diskretisierungsprobleme zu beachten.
Selbstverschattung lässt sich in Szenen mit solider Geometrie am besten mit Second-Depth Shadow Mapping bekämpfen.
In Szenen ohne solider Geometrie kann Selbstverschattung mit einer Kombination aus Biasing Methoden verhindert werden.
Die leichteste Form Schatten Aliasing zu verhindern, ist das Anwenden von Antialiasing Filtering wie PCF.
Perspektivisches Aliasing führt jedoch oft zu Situationen, in denen Filtering nicht weiter hilft.
In solchen Fällen sind Warping Methoden wie PSM oder LisPSM zwar eine gute Option sind aber in der Praxis oft schwer zu implementieren.
CSM stellt eine Diskretisierung der Warping Methoden dar und ist dabei sehr viel leichter zu implementieren da hier die Schattenberechnung in 
World Space durchgeführt wird.
\par
CSM + PCF + Second Depth Shadow Mapping, ist eine größtenteils echtzeitfähige Implementierung.
Second Depth Shadow Mapping hat keinen Einfluss auf die Performance und kann Selbstverschattung vollständig verhindern.
CSM hat einen negativen Einfluss auf die Performance.
Dieses ist jedoch in den meisten Fällen akzeptabel, da die Schattenqualität von CSM oft besser ist.
Im Bereich des Schattenkanten Antialiasing ist jedoch der Performance Unterschied zwischen PCF und alternativen 
Antialiasing Methoden nicht akzeptabel.
Auch wenn diese Kombination auf heutiger Hardware echtzeitfähig ist, 
sollte erwähnt werden, dass PCF kein performanter Algorithmus ist. 
Es gibt jedoch eine große Menge an Algorithmen wie Variance Shadow Mapping oder Exponential Shadow Mapping die 
eine performante Alternative zu PCF darstellen.





















%% --------------------
%% |   Bibliography   |
%% --------------------
\cleardoublepage
\phantomsection
\addcontentsline{toc}{chapter}{\bibname}

\iflanguage{english}
{\bibliographystyle{IEEEtranSA}}	% english style
{\bibliographystyle{babalpha-fl}}	% german style
												  
% Use IEEEtran for numeric references
%\bibliographystyle{IEEEtranSA})


\bibliography{ausarbeitung}
%\Erklaerung
\end{document}
