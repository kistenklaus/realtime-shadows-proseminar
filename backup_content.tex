%% content.tex
%%


\chapter{Warum sind Schatten eigentlich so wichtig?}
\chapter{Geschichte}
\chapter{Shadow Mapping}
\chapter{Visuelle Artefakte}
\section{Schatten Akne}
\section{Perspektivisches Aliasing}
\section{Projektives Aliasing}
\chapter{Schatten bias}
\section{Peter Panning}
\section{Slope Scale Bias}
\section{Normal Offset Bias}
\section{Second-depth Shadow Mapping}
\section{Fazit}
\chapter{Perspektivisches Shadow Mapping}
\section{LiPSM}
\section{Cascaded Shadow Mapping}
\chapter{Punkt Lichtquellen}


%% ==============================

\chapter{Einleitung}
\label{ch:Introduction}
Eine Einleitung beginnt üblicherweise mit einer Motivation für das Thema und gegebenenfalls einer thematischen Einordnung.
Im Anschluss folgt eine kurze Zusammenfassung der gesamten schriftlichen Arbeit.
Abschließend folgt häufig ein Absatz, in dem die Kapitelgliederung der Arbeit erläutert wird, z.B:

In der folgenden Arbeit werden zunächst die nötigen Grundlagen für die verschiedenen Techniken aufgezählt (Kapitel \ref{ch:Content1}).
Anschließend werden \dots



\vfill
Hier noch ein Tipp für das Arbeiten mit git/Versionskontrollsoftware und Latex:
Wenn jeder Satz im \texttt{.tex}-Dokument in einer eigenen Zeile steht, sind die commits etwas aufgeräumter und Merging funktioniert besser.


%% ==============
\chapter{Ein Kapitel}
\label{ch:Content1}


\section{Erster Abschnitt}
Eine Referenz sieht so aus \cite{becker2008a}.
Alles, was aus anderen Quellen übernommen wurde, muss auf diese Weise zitiert werden.
Alle nicht-trivialen Aussagen sollten mit einer Quelle belegt werden, sofern sie nicht eigens hergeleitet werden.

%
Und so referenzieren wir ein Bild (Abbildung \ref{fig:beispielbild}).
Jede Abbildung sollte einen Zweck im Text erfüllen.
Konkret heißt das: Jede Abbildung sollte an irgendeiner Stelle im Text referenziert werden.

\begin{figure}
	\begin{center}
		\includegraphics[width=.3\textwidth]{logos/KITLogo_RGB.pdf}
		\caption{Ein Beispielbild. Abbildungsunterschriften sollten die Abbildung knapp erklären und enden immer mit einem Punkt.
			Bilder können aus anderen Quellen übernommen werden.
			Auch das muss gekennzeichnet werden, z.B. so: Abbildung aus \cite{becker2008a}.}
	\end{center}
	\label{fig:beispielbild}
\end{figure}


\section{Zweiter Abschnitt}
Dies ist ein langer Text, der dafür sorgt, dass alsbald ein Zeilenumbruch erfolgt: $x$"~Koordinatensystem.
Lorem ipsum dolor sit amet, consetetur sadipscing elitr, sed diam nonumy eirmod tempor invidunt ut labore et dolore magna aliquyam erat, sed diam voluptua.
At vero eos et accusam et justo duo dolores et ea rebum.
Stet clita kasd gubergren, no sea takimata sanctus est Lorem ipsum dolor sit amet.
Lorem ipsum dolor sit amet, consetetur sadipscing elitr, sed diam nonumy eirmod tempor invidunt ut labore et dolore magna aliquyam erat, sed diam voluptua.
At vero eos et accusam et justo duo dolores et ea rebum. Stet clita kasd gubergren, no sea takimata sanctus est Lorem ipsum dolor sit amet.

\dots


%% content.tex
%%

%% ==============
\chapter{Ein Kapitel}
\label{ch:Content2}
%% ==============

\dots


%% ===========================
\section{Erster Abschnitt}
\label{ch:Content2:sec:Section1}
%% ===========================

Mathematische Gleichungen wie $x = y + 1$ können im Text mit \$ \dots \$ genutzt werden.
Längere oder wichtigere Gleichungen wie Gleichung \ref{eq:mittelwert} können als separate Objekte gesetzt werden und referenziert werden.
Dabei sind auch diese Gleichungen als Teil des Textes anzusehen und mit den nötigen Satzzeichen auszustatten.
Das arithmetische Mittel ist definiert als
\begin{equation}
\frac{1}{n} \sum_{i=1}^{n} x_i \; ,
\label{eq:mittelwert}
\end{equation}
wobei $n$ die Anzahl der Elemente ist.
Gleichungen, die nicht im Text referenziert werden, benötigen keine Nummerierung.
Das lässt sich mit der Environment \texttt{equation*} umsetzen.
Das geometrische Mittel ist
\begin{equation*}
	\sqrt[n]{\sum_{i=1}^{n} x_i} \; .
\end{equation*}





%% ===========================
\section{Zweiter Abschnitt}
\label{ch:Content2:sec:Section2}
%% ===========================

\dots
